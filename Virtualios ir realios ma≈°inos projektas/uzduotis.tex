\section{Užduotis}
\textbf{Projektuojama paketinė OS}

Virtualios mašinos procesoriaus komandos operuoja su duomenimis, esančiais registruose ir ar atmintyje. Yra komandos duomenų persiuntimui iš atminties į registrus ir atvirkščiai, aritmetinės (sudėties, atimties, palyginimo), sąlyginio ir besąlyginio valdymo perdavimo, įvedimo, išvedimo, darbo su bendra atminties sritimi (prieinama visoms vartotojo programoms; komandos leidžia į ją rašyti ir skaityti; sritis apsaugoma semaforais) ir programos pabaigos komandos. Registrai yra tokie: komandų skaitiklis, bent du bendrosios paskirties registrai, požymių registras (požymius formuoja aritmetinės, o į juos reaguoja sąlyginio valdymo perdavimo komandos). Atminties dydis yra 16 blokų po 16 žodžių (žodžio ilgį pasirinkite patys).

Realios mašinos procesorius gali dirbti dviem režimais: vartotojo ir supervizoriaus. Virtualios mašinos atmintis atvaizduojama į vartotojo atmintį naudojant puslapių transliaciją. Yra taimeris, kas tam tikrą laiko intervalą generuojantis pertraukimus. Įvedimui naudojamas virtualių „flash atmintinių“ nuskaitymo įrenginys, išvedimui - spausdintuvas. Yra išorinės atminties įrenginys - kietasis diskas.

Vartotojas užduočių paketą pateikia „prijungęs“ atmintinę. Sistema perkelia visas joje esančias užduotis į diską, patikrindama jų sintaksę, ir, tuo pat metu, jei tik yra reikiamų resursų, pradeda jas vykdyti.


