\section{Virtuali mašina operacinės sistemos kontekste}

Multiprograminėje operacinėje sistemoje centrinis procesorius yra perjunginėjamas iš vieno proceso (vykdomos programos) į kitą, sukuriant lygiagretaus veikimo įspūdį. Apibrėžiant šio mechanizmo sąvoką, naudojami proceso, resurso, proceso paketo terminai.

Kiekvienas procesas turi savo virtualios mašinos procesorių. Kad procesas būtų vykdomas, jis turi gauti procesoriaus resursų. 

Taigi virtuali mašina visos operacinės sistemos kontekste atlieka labai svarbų vaidmenį – ji reikalinga proceso vykdymui.


