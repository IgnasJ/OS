\section{Virtuali mašina operacinės sistemos kontekste}

Virtuali mašina yra realios kopija, kuri veikia kaip tarpininkas. Viruali mašina paslepia realios mašinos realizaciją po virtualiais komponentais. Ji turi savo atmintį, kur kiekvienas blokas turi tiek virtualų tiek realų adresą. Ryšiai tarp šių adresų nusakomi puslapių lentelėmis. Virtuali mašina supaprastina vartotojo sąsają ir vykdo programą, kuri yra virtualioje atmintyje.
Taigi OS – tai programa, kuri modeliuoja kelių virtualių mašinų darbą vienoje realioje mašinoje.

\newpage


